\documentclass[11pt]{article}
\usepackage{amsmath}
\usepackage{listings}
\usepackage{geometry}
\usepackage{hyperref}
\usepackage{fancyhdr}
\usepackage{titlesec}
\usepackage{xcolor}

% Page layout
\geometry{letterpaper, margin=1in}
\setlength{\headheight}{15pt}

% Fancy header and footer
\pagestyle{fancy}
\fancyhf{}
\fancyhead[L]{Trust or Treachery Tournament}
\fancyhead[R]{Fusion Academy - The Woodlands}
\fancyfoot[C]{\thepage}

% Title formatting
\titleformat{\section}{\normalfont\Large\bfseries}{\thesection}{1em}{}
\titleformat{\subsection}{\normalfont\large\bfseries}{\thesubsection}{1em}{}

% Code formatting
\lstdefinestyle{mystyle}{
    backgroundcolor=\color{white},
    basicstyle=\ttfamily\footnotesize\color{black},
    commentstyle=\color{gray},
    keywordstyle=\color{blue},
    stringstyle=\color{red},
    numberstyle=\tiny\color{gray},
    identifierstyle=\color{black},
    breaklines=true,
    frame=single,
    columns=fullflexible
}
\lstset{style=mystyle}

\title{Trust or Treachery Tournament}
\author{Fusion Academy - The Woodlands}
\date{}

\begin{document}

\maketitle

\section*{Overview}
Welcome to the end-of-year Trust or Treachery Tournament! In this project, you will write a Python function that implements a strategy for a special variant of the prisoner's dilemma game. Your function will compete against those of your classmates over a series of rounds. The goal is to accumulate the highest total score based on the payoff matrix. \textbf{Critical Thought: }Does this mean you should also aim to minimize your opponent's payout?

\section*{Game Description}
The prisoner's dilemma is a classic game theory problem. Two players simultaneously choose to either \textbf{cooperate (C)} or \textbf{defect (D)}. The payoffs for each player are determined as follows:
\begin{itemize}
    \item Both cooperate: Each player receives 3 points.
    \item Both defect: Each player receives 1 point.
    \item One cooperates, one defects: The cooperator receives 0 points, and the defector receives 5 points.
\end{itemize}

The dilemma arises because mutual cooperation yields the best collective outcome, but individual incentives can lead to defection, resulting in a worse outcome for both players. The game illustrates the conflict between individual rationality and collective rationality.

\section*{Enhanced Game Description}
To incorporate a real-world phenomenon, we will introduce the possibility of misinterpreting the actions of others. In each round, there will be a 2\% chance that a player's intended action will be flipped. This simulates the occasional misunderstanding or miscommunication that can occur in real-life situations.

\subsection*{Noise/Static in Decisions}
\begin{itemize}
    \item There is a 2\% chance that the chosen action (cooperate or defect) will be flipped due to external factors.
\end{itemize}

\subsection*{Number of Rounds}
\begin{itemize}
    \item Each match will consist of 195 to 205 rounds, randomly selected at the start of each match.
\end{itemize}

\subsection*{Dynamic Payoff Matrix}
\begin{itemize}
    \item Initially, the payoff matrix is as follows:
    \begin{itemize}
        \item Both cooperate: Each player receives 3 points.
        \item Both defect: Each player receives 1 point.
        \item One cooperates, one defects: The cooperator receives 0 points, and the defector receives 5 points.
    \end{itemize}
    \item After 100 rounds of mutual cooperation, the payoff matrix changes to:
    \begin{itemize}
        \item Both cooperate: Each player receives 4 points.
        \item Both defect: Each player receives 1 point.
        \item One cooperates, one defects: The cooperator receives 0 points, and the defector receives 10 points.
    \end{itemize}
\end{itemize}

\subsection*{Nuclear Option}
To add another layer of strategy, we introduce the nuclear option. A player can choose to stop the match with the current score and walk away. This option can be useful when dealing with an untrustworthy opponent. 

\begin{itemize}
    \item If one player invokes the nuclear option (returns 'N'), the match ends immediately.
    \item The player who invokes the nuclear option receives a +5 reputation bonus, while the opponent gets a -5 reputation penalty.
    \item If both players invoke the nuclear option simultaneously, the match ends immediately, and no reputation changes are made.
\end{itemize}

\section*{Function Signature}
You will need to implement a function with the following signature:
\begin{lstlisting}[language=Python]
def my_prisoner_strategy(my_history: list[str], 
                         opponent_history: list[str], 
                         round_number: int,
                         relative_score: int,
                         opponent_reputation: int) -> str:
    """
    Determines whether to cooperate, defect, or use the nuclear option.

    Args:
        my_history (list[str]): A list of 'C' (cooperate) or 'D' (defect) representing my past moves.
        opponent_history (list[str]): A list of 'C' or 'D' representing the opponent's past moves.
        round_number (int): The current round number.
        relative_score (int): The difference between the player's score and the opponent's score.
        opponent_reputation (int): The opponent's reputation score.

    Returns:
        str: 'C' for cooperate, 'D' for defect, or 'N' for nuclear option.
    """
    pass
\end{lstlisting}

\section*{Example Strategies}
Here are some example strategies to get you thinking:

\subsection*{Always Cooperate}
\begin{lstlisting}[language=Python]
def always_cooperate(my_history, opponent_history, round_number, relative_score, opponent_reputation):
    return 'C'
\end{lstlisting}

\subsection*{Always Defect}
\begin{lstlisting}[language=Python]
def always_defect(my_history, opponent_history, round_number, relative_score, opponent_reputation):
    return 'D'
\end{lstlisting}

\subsection*{Tit for Tat}
\begin{lstlisting}[language=Python]
def tit_for_tat(my_history, opponent_history, round_number, relative_score, opponent_reputation):
    if round_number == 1:
        return 'C'
    else:
        return opponent_history[-1]
\end{lstlisting}

\subsection*{Score-Based Strategy}
\begin{lstlisting}[language=Python]
def score_based_strategy(my_history, opponent_history, round_number, relative_score, opponent_reputation):
    if round_number == 1:
        return 'C'
    if relative_score > 0:
        return 'C'
    return 'D'
\end{lstlisting}

\subsection*{Reputation-Based Strategy}
\begin{lstlisting}[language=Python]
def reputation_based_strategy(my_history, opponent_history, round_number, relative_score, opponent_reputation):
    if round_number == 1:
        return 'C'
    if opponent_reputation < -10:
        return 'N'
    if relative_score > 0:
        return 'C'
    return 'D'
\end{lstlisting}

\subsection*{Nuclear Option Strategy}
\begin{lstlisting}[language=Python]
def nuclear_option_strategy(my_history, opponent_history, round_number, relative_score, opponent_reputation):
    if round_number == 1:
        return 'C'
    if opponent_history[-3:] == ['D', 'D', 'D']:
        return 'N'
    return 'C' if relative_score > 0 else 'D'
\end{lstlisting}

\section*{Scoring and Leaderboard}
Each match will be scored based on the payoff matrix. Your total score from all matches will determine your ranking on the leaderboard. The leaderboard will be updated in real-time, so you can track your progress throughout the tournament.

\section*{Guidelines}
\begin{itemize}
    \item Your function must only use the provided arguments (\texttt{my\_history}, \texttt{opponent\_history}, \texttt{round\_number}, \texttt{relative\_score}, \texttt{opponent\_reputation}) to make its decision.
    \item You must document your strategy, explaining your thought process and expected outcomes.
    \item Be creative and think critically about how to outsmart other strategies.
\end{itemize}

\section*{Implementation Details}
The following code snippets outline the main components of the tournament platform:

\subsection*{Dynamic Payoff Matrix}
\begin{lstlisting}[language=Python]
initial_payoff_matrix = {'CC': (3, 3), 'CD': (0, 5), 'DC': (5, 0), 'DD': (1, 1)}
advanced_payoff_matrix = {'CC': (4, 4), 'CD': (0, 10), 'DC': (10, 0), 'DD': (1, 1)}

def get_payoff_matrix(mutual_cooperations):
    """
    Determine the appropriate payoff matrix based on the number of mutual cooperations.

    Args:
        mutual_cooperations (int): The number of rounds both players cooperated.

    Returns:
        dict: The current payoff matrix.
    """
    if mutual_cooperations >= 100:
        return advanced_payoff_matrix
    return initial_payoff_matrix
\end{lstlisting}

\subsection*{Noise/Static in Decisions}
\begin{lstlisting}[language=Python]
import random

def add_noise(action: str, noise_level: float = 0.02) -> str:
    """
    Introduce noise to the chosen action with a given probability.

    Args:
        action (str): The intended action ('C' or 'D').
        noise_level (float): Probability of action being toggled.

    Returns:
        str: The possibly toggled action.
    """
    if random.random() < noise_level:
        return 'D' if action == 'C' else 'C'
    return action
\end{lstlisting}

\newpage

\subsection*{Play Match Function with Static Noise}
\begin{lstlisting}[language=Python]
def play_match(strategy1, strategy2, rounds):
    history1, history2 = [], []
    score1, score2 = 0, 0
    reputation1, reputation2 = 0, 0
    mutual_cooperations = 0
    noise_level = 0.02 

    for round_number in range(1, rounds + 1):
        payoff_matrix = get_payoff_matrix(mutual_cooperations)
        relative_score1 = score1 - score2
        relative_score2 = score2 - score1
        
        move1 = strategy1(history1, history2, round_number, relative_score1, reputation2)
        move2 = strategy2(history2, history1, round_number, relative_score2, reputation1)

        if move1 == 'N' and move2 == 'N':
            # Both choose nuclear option, end the match without reputation change
            break
        elif move1 == 'N':
            # Only player 1 chooses nuclear option, update reputation
            reputation1 += 5
            reputation2 -= 5
            break
        elif move2 == 'N':
            # Only player 2 chooses nuclear option, update reputation
            reputation2 += 5
            reputation1 -= 5
            break

        move1 = add_noise(move1, noise_level)
        move2 = add_noise(move2, noise_level)

        outcome = move1 + move2
        if outcome == 'CC':
            mutual_cooperations += 1
        score1 += payoff_matrix[outcome][0]
        score2 += payoff_matrix[outcome][1]

        history1.append(move1)
        history2.append(move2)

        # Update reputations
        reputation1 = calculate_reputation(history1, history2)
        reputation2 = calculate_reputation(history2, history1)

    return score1, score2, reputation1, reputation2
\end{lstlisting}

\newpage

\subsection*{Tournament Execution}
\begin{lstlisting}[language=Python]
def run_tournament(strategies):
    results = {strategy.__name__: {'score': 0, 'reputation': 0} for strategy in strategies}

    for i, strategy1 in enumerate(strategies):
        for j, strategy2 in enumerate(strategies):
            if i != j:
                rounds = random.randint(195, 205)
                score1, score2, reputation1, reputation2 = play_match(strategy1, strategy2, rounds)
                results[strategy1.__name__]['score'] += score1
                results[strategy1.__name__]['reputation'] = reputation1
                results[strategy2.__name__]['score'] += score2
                results[strategy2.__name__]['reputation'] = reputation2

    return results
\end{lstlisting}

We hope you enjoy this project and learn a lot about game theory and strategic thinking. Good luck!

\end{document}
